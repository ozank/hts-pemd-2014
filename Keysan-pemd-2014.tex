\documentclass[12pt]{IET02}
\usepackage{amsmath,amssymb}

%% Uncomment below line (note epsf package) and comment 
%% above line if using EPS figures - then use LaTeX --> DVI to PDF.
%% else, the template compiles straight using PDFLaTeX --> PDF
%\usepackage{amsmath,amssymb,epsf}

%% Load graphicx package
\usepackage{mathptmx,graphicx}

\begin{document}

\title{Towards More Reliable and Cost Effective \\Superconducting Generators for Wind Turbines}

\markboth{Author title}{Article title}

\author{O. Keysan$^{*}$, P. Radyjowski, J. Burchell, M. A. Mueller}

\address{\textit{Institute for Energy Systems, University of Edinburgh, United Kingdom}\\
$^{*}$\textit{Email: o.keysan@ed.ac.uk}}

\keyword{superconducting generators, direct-drive, offshore wind turbines}

%\begin{abstract}
%\end{abstract}

\twocolumn

\maketitle

\section*{Abstract}
Larger offshore wind turbines help to reduce the installation and maintenance cost per unit energy generated. However, tower-head mass becomes a real challenge with conventional power take-off systems as the power rating is increased. Direct-drive superconducting generators are proposed to overcome this issue. There are many 10 MW, 10 rpm superconducting generator designs, most of which having the same topology: a synchronous machine with a conventional copper armature winding and a superconducting rotor. However, this design may not be the most suitable topology for an offshore wind turbine application, where the O&M costs are very high. 
The rotating superconducting field windings in this type of machines require rotating transfer couplings to transfer the cooling gas and they also need brushes or brushless exciters, which remains as the single point of failure in the machine, decreasing the overall reliability. Furthermore, in these designs the electromagnetic forces act on the superconducting windings, which makes the structural and thermal design challenging.
The cost of the power take-off system for a wind turbine is also very critical, and the cost of a superconducting machine is usually dominated by the cost of superconducting tape. In the common designs, the superconducting tape requirement is usually in the order of hundreds of kilometres because of two main factors: the superconducting field winding can be air-cored, which increases the MMF requirements. Secondly, even if the machine is iron-cored, the main flux path has to cross the insulation layers, vacuum and cryostat walls, which increases the equivalent magnetic gap, even if the mechanical gap is kept minimum.
In this paper, a novel transverse flux superconducting machine topology that can overcome these issues will be presented. The machine has stationary superconducting field windings and a modular claw-pole rotor. The field winding is also modular and can be placed in independent cryostat sections. Thus, even in one of the cryostat section fails machine, the machine can still operate at partial load until the maintenance. Having smaller cryostat sections also have logistics advantages such as easy installation/ transportation, and using on-site crane for any replacements. Another advantage of the topology is the minimum superconducting wire requirement. A 10 MW design is estimated to use just 15 km of MgB2 wire at 30 K. This is due to iron-cored structure and because main flux path does not cross through the cryostat (i.e. magnetic gap equals to the mechanical gap) which minimizes the required MMF.
A small linear prototype with copper winding is manufactured to prove the concept and it is planned to replace the copper winding with a high-temperature superconducting winding and repeat the tests with a proper superconducting coil at 30 K. The results will be presented in the full paper.
\vspace{1pc}

\section{Introduction}

Superconducting generators known 1960s
low temperature HTS generator

Ship propulsion systems (36.5 MW generator)

Large offshore wind turbines
- 10 MW
- tower head mass
- direct drive generators


\section{A Review of Superconducting Machines}

literaturdeki wind turbinler

most common type: HTS rotor synchronous

magnetised generators

electromagnetic pumping 

\section{Requirements of a Wind Turbine Generator}


\section{Manuscript preparation}

Full papers must be typed in English. This instruction page is an
example of the format and font sizes to be used.

The title of the paper is typed in sentence case (bold 18pt) and
centred on the page. The author's initials and surname are typed
in title case bold letters and centred on the page. Directly under
the author's name in title case letters and also centred is the
author's affiliation, address, plus email address of (at least)
the corresponding author. Manuscripts must be typed single spaced
using 10 point characters. Only Times, Times Roman, Times New
Roman and Symbol fonts are accepted. The text must fall within a
frame of $18\,\hbox{cm} \times 24\,\hbox{cm}$ centred on an A4
page $(21\,\hbox{cm} \times 29.7\,\hbox{cm})$. Paragraphs are
separated by 6 points and with no indentation. The text of the
full papers is written in two columns and justified. Each column
has a width of 8.8\,cm and the columns are separated by a margin
of 0.4\,cm. Your paper must adhere to the length stipulated on the
event website. Pages should be numbered centrally at the bottom of
the page.

\subsection{\it Figures and tables}

Figures and tables should be centred in the column, numbered
consecutively throughout the text, and each should have a caption
underneath it (see for example Table 1). Care should be taken that
the lettering is not too small. All figures and tables should be
included in the electronic versions of the full paper.

\begin{table}[h]%1
\processtable{This is an example of a table caption}
{\begin{tabular}{@{}|@{\ \ \qquad}l@{\ \ \qquad}|@{\ \ \qquad}c@{\ \ \qquad}|@{}}\hline
 &\\[-.6pc]
$n$ &$n$! \\[-.6pc]
 &\\\hline
 &\\[-.6pc]
1 &1\\[-.6pc]
 &\\\hline
 &\\[-.6pc]
2 &2\\[-.6pc]
 &\\\hline
 &\\[-.6pc]
3 &3\\[.2pc]\hline
\end{tabular}}{}
\end{table}

\subsection{\it Equations}

Equations should be typed within the text, centred, and should be numbered
consecutively throughout the text. They should be referred to in the text as
Equation~(n). Their numbers should be typed in parentheses, flush right, as
in the following example.
\begin{equation}
PA + A'P - PBR^{-1}B'P + Q = 0.
\end{equation}

\section{Final PDF file}

The final format in which the papers will appear in the Proceedings will be
a PDF file. Authors are requested to send a PDF file of their final paper to
be included directly in the Proceedings. Do not lock your PDF file -- this
may result in your paper not being published.

\section{Submission of the full paper}

Your full paper should be submitted online via the conference website. It
should be expected that after your submission, your final paper will be
published directly from the PDF you send without any further proof-reading.
Therefore, it is advisable for the authors to print a hard copy of their
final version and read it carefully.

\section*{Acknowledgements}

The acknowledgement for funding organisations etc. should be placed in a
separate section at the end of the text.

Thank you for your cooperation in complying with these instructions.

\begin{thebibliography}{9}
\vspace{1pc}

\bibitem{}A. B. Author, C. D. Author. ``Title of the article'',
\textit{The Journal}, \textbf{volume}, pp.~110--120, (2000).\vspace{.4pc}
\end{thebibliography}

The reference list should be in numerical order and each number matches and
refers to the one in the text. All references should be cited in the text,
and using square brackets such as [1] and [2, 3].

\end{document}
